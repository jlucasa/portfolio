% Use the custom resume.cls style
\documentclass{resume} 

% Document margins
\usepackage[left=0.75in,    top=0.4in,  right=0.75in,   bottom=0.3in]{geometry} 

% Stuff for setting default indentation for new paragraphs
\newcommand{\tab}[1]{\hspace{.2667\textwidth}\rlap{#1}}
\newcommand{\itab}[1]{\hspace{0em}\rlap{#1}}

\usepackage{enumitem}
\usepackage{changepage}
\usepackage{parskip}
\usepackage{multicol}

\usepackage{hyperref}
\hypersetup{
    colorlinks=true,    % Flag for whether links should be a different color
    linkcolor=blue,     % Color of in-page links (not used)
    filecolor=magenta,  % Color of local filesystem links (not used)
    urlcolor=blue,      % Color of web links (DEFINITELY USED!)
}

%------------------------------------------
%
%   CONTACT INFO
%
%------------------------------------------
\name{Jared Amen} 
\address{Salt Lake City, UT 84111 \\
         (208)-705-5146}
\address{
    \href{mailto:amen.jaredl@gmail.com}
         {amen.jaredl@gmail.com} \\ 
    \href{https://linkedin.com/in/jared-amen}{linkedin.com/in/jared-amen} \\ 
    \href{https://github.com/jlucasa}{github.com/jlucasa}
}    

\begin{document}

%---------------------------------------------------------------------------------------
%
%	SUMMARY
%
%----------------------------------------------------------------------------------------

% \begin{center}
%     \vspace{-5pt}
%     {\em Engineer solutions with creative approaches, quantitative insights, and intuitive software design}
%     \vspace{-2pt}
% \end{center}

%----------------------------------------------------------------------------------------
%
%	EDUCATION
%
%----------------------------------------------------------------------------------------

\begin{rSection}{Education}

% TODO: Edit resume.cls to handle formatting better.
{\bf University of Utah} \hfill {\em GPA: 3.67, Graduated May 2022}
\vspace{2pt}
\emph{
    \\ Bachelor's \& Master's of Science in Computer Science -- Human-Centered Computing Emphasis
    \\ Teaching Assistant (2021-2022), Vice President of Triangle Engineering Fraternity (2021-2022)
}

\vspace{-0.3em}
{\bf Skills:}
\vspace{-1.83em}

\begin{adjustwidth}{3.5em}{0pt}
    {\bf Algorithms},
    {\bf Data Structures},
    {\bf Human-Computer Interaction},
    Deep Learning,
    Natural Language Processing,
    Databases,
    Data Analytics,
    Large-Scale Software Applications,
    Object-Oriented Design
\end{adjustwidth}

\vspace{-0.4em}
{\bf Languages:}
\vspace{-1.83em}
\begin{adjustwidth}{6em}{0pt}
    {\bf \href{https://github.com/jlucasa?tab=repositories&q=&type=&language=python&sort=}{Python}}, 
    {\bf \href{https://github.com/jlucasa?tab=repositories&q=&type=&language=javascript&sort=}{JavaScript}},
    \href{https://github.com/jlucasa?tab=repositories&q=&type=&language=tex&sort=}{\LaTeX},
    TypeScript,
    Java,
    C,
    C++,
    C\#,
%    Visual Basic,
    Ruby
\end{adjustwidth}

\vspace{-3pt}
{\bf Technologies:}
\vspace{-1.83em}
\begin{adjustwidth}{7em}{0pt}
    \href{https://github.com/jlucasa?tab=repositories&q=&type=&language=jupyter+notebook&sort=}{Jupyter Notebook},
    Git,
    React,
%    Ubuntu,
    macOS,
    Windows,
    bash/zsh/Powershell,
    Angular,
    Ruby on Rails,
    Node.js,
    GraphQL,
    D3.js,
    Keras,
    Google Cloud Platform,
    AWS (S3, ECS),
    MongoDB

\end{adjustwidth}

\end{rSection}

%----------------------------------------------------------------------------------------
%
%   EXPERIENCE 
%   ----------
%
%   XYZ FORMATTING:
%
%   Performed [X], resulting in [Y], using [Z].
%
%   X: What you did
%   Y: Why you did it
%   Z: How you did it
%
%   X and Z are the most important bits - Y is nice if you have quantifiable evidence
%   of the result of your action.
%
%----------------------------------------------------------------------------------------

\begin{rSection}{Work Experience}
    {\bf Software Engineering Intern, R\&D}, {\em \href{https://www.kantata.com/}{Kantata} (formerly Mavenlink) \hfill Summer 2022}
    \vspace{-6pt}
    \begin{itemize}[nosep]
        \item Resolved $\approx$30 direct support requests for the administrator panel of Kantata's production application to allow new functionality to fetch and filter through millions of account logs, manage custom subdomains, and perform bulk actions on thousands of accounts.
        \item Wrote 10 pages of detailed, step-by-step in-house documentation for Kantata engineers to use in making configurable tables using their own data provided via GraphQL, lowering the time to create these tables to a scale of $\approx$1 business day.
    \end{itemize}
    
    % {\bf Teaching Assistant}, {\em University of Utah}
    % \vspace{-6pt}
    % \begin{adjustwidth}{1em}{0pt}
    %     {\bf CS 4400} {\em (Computer Systems, Prof. Ben Jones) \hfill Jan 2022 - May 2022}\\
    %     {\bf CS 1420} {\em (Accel. Intro to Object-Oriented Programming, Prof. Peter Jensen) \hfill Aug 2021 - Dec 2021}
    % \end{adjustwidth}
    % \vspace{-3pt}
    % \begin{itemize}[nosep]
    %     \item Leveraged skills in C/x86 for CS 4400 and Java for CS 1420 to revise and grade assignments/exams.
    %     \item Held labs/office hours for students, helping students outside of office hours through Piazza.
    % \end{itemize}
    
    % {\bf Teaching Assistant}, {\em CS 4400 (Computer Systems), University of Utah \hfill Jan 2022 - May 2022}
    % \vspace{-6pt}
    % \begin{itemize}[nosep]
    %     \item Leveraged skills in C and x86 to revise/grade assignments/exams and hold labs/office hours for students, helping students outside of office hours through Piazza.
    % \end{itemize}
    
    % {\bf Teaching Assistant}, {\em CS 1420 (Accel. Intro to OOP), University of Utah \hfill Aug 2021 - Dec 2021}
    % \vspace{-6pt}
    % \begin{itemize}[nosep]
    %     \item  Taught an accelerated introduction to object-oriented programming in Java to first-year students through a combination of labs and one-on-one assistance through office hours and Piazza.
    % \end{itemize}
    
    {\bf Project Lead}, {\em \href{https://civilmedia.webflow.io/}{Civil Media} \hfill Oct 2020 - Sep 2021}
    \vspace{-6pt}
    \begin{itemize}[nosep]
        \item Lead a team of three to design a website with structured, balanced, interactive modules on controversial topics, used by K-12 educators across the US to improve lesson plans on these topics.
        \item Design and improve upon an AGILE approach to project management, which improved time engagement as recorded by Jira by $\approx$5 hours per member per week.
    \end{itemize}

    % {\bf Software Engineering Intern}, {\em University of Utah \hfill Jul 2019 - Apr 2020, Aug 2020 - Aug 2021}
    % \vspace{-6pt}
    % \begin{itemize}[nosep]
    %     \item Worked in a team of four to develop a new C\# backend and React frontend for the interfacing tool that the university's health departments use to access student/staff/faculty data, resulting in $\approx$2x increased speed when using the web app.
    % \end{itemize}

    {\bf Software Engineering Intern}, {\em \href{https://www.instructure.com/}{Instructure} \hfill Summer 2020}
    \vspace{-6pt}
    \begin{itemize}[nosep]
        \item Used A11y guidelines to provide screen readers compatible with JAWS and VoiceOver for Canvas, an LMS used by 4,000 institutions across the world.
        \item Developed a new rich-content editor powered by \href{https://www.tiny.cloud/}{TinyMCE}, using in-house plugins written in Ruby on Rails and React for media uploading, allowing for existing database setups to remain.
        % \item Built a {\bf Kubernetes/Docker} PaaS to simplify deployment of science apps on distributed systems
        % \item Iteratively redesigned project site in {\bf React.js, HTML \& CSS} and monitor Google Analytics for site
        %\item Researched the use of provisioning software, such as {\bf Foreman}, to provision new server clusters
    \end{itemize}
    
    % {\bf Undergraduate Research Assistant}, {\em University of Utah} \hfill {\em October 2019 - January 2020}
    % \vspace{1pt}
    % \\ Advisor: Dr. Blair D. Sullivan, Research Team of 4
    % \vspace{-6pt}
    % \begin{itemize}[nosep]
    %     \item Provided optimized C++ translations from Python for a way to ``turbocharge'' odd cycle transversal algorithms on a graph with fixed parameter tractability
    %     \item Achieved an approximate 25\% performance boost over Python implementation.
    % \end{itemize}
    
\end{rSection}

%-----------------------------------------------------------------------------------------------
%
%   PROJECTS
%
%-----------------------------------------------------------------------------------------------
\begin{rSection}{Notable Projects}

    \href{https://github.com/jlucasa/style-transfer-vis}{{\bf Style Transfer Workbench}}, {\em Team of 3} {\em \hfill Dec 2021}
    \vspace{-6pt}
    \begin{itemize}[nosep]
        \item Developed a machine learning workbench prototype for visually interpreting and exploring activation values in the style transfer problem using SqueezeNet Neural Network Architecture
        \item Specialized in user interface, utilizing Streamlit and Google Colab to provide a clean user experience superpowered by GPUs making the style transfer process tractable.
    \end{itemize}

    % \href{https://github.com/No-Losing-Parses/src}{{\bf Question/Answering System}}, {\em Team of 2, 1st place out of $\approx$30 teams} {\em \hfill Nov 2020}
    % \vspace{-6pt}
    % \begin{itemize}[nosep]
    %     \item Created a single-document question/answering system used on news articles for a graduate-level natural language processing class competition at the University of Utah
    %     \item Achieved performance boosts of $\approx$10\% through enhanced phrase extraction that leveraged the cosine similarity between the embeddings of tokens in questions and tokens in potential answers
    % \end{itemize}
    
    {\bf \href{https://bdv-2021.web.app}{Bank Document Verifier}}, {\em Team of 3, 2nd place out of $\approx$40 teams} {\em \hfill Sep 2020 - May 2021}
    \vspace{-6pt}
    \begin{itemize}[nosep]
        \item Created a web program in collaboration with \href{https://enerbank.com/}{Enerbank USA} for the automatic detection of fraud in applications for loans from financial institutions
        \item Specialized in a comprehensive user management panel for administrators of the program, and created broad security protocols that fit Enerbank's requirements, reducing the cost-of-use by $\approx$75\% by only issuing API calls when necessary
    \end{itemize}
    
    % {\bf \href{https://github.com/jlucasa/link-waiter}{Link-Based ``Finder'' Discord Bot}} {\em \hfill September 2020 - Current}
    % \vspace{-6pt}
    % \begin{itemize}[nosep]
    %     \item Created in discord.py for my senior project group's Discord server
    %     \item Utilizes AWS S3 to store and retrieve links in simple JSON files for the future, and has improved project workflow in the group
    % \end{itemize}
    
    % \href{https://devpost.com/software/beethoven-t9ud86}{\bf Beethoven}, {\em 2nd place out of $\approx$30 teams
    % \hfill HackTheU 2019}
    % \vspace{-6pt}
    % \begin{itemize}[nosep]
    %     \item Designed a closed captioning and audio transcription service for deaf and hard-of-hearing students
    %     \item Created a {\bf Node.js \& React} web application for peer-to-peer text \& audio streaming via {\bf WebRTC}
    % \end{itemize}
    
    % \href{https://www.linkedin.com/feed/update/urn:li:activity:6603722406240628736/}{\bf LED Music Visualizer}
    % \vspace{-6pt}
    % \begin{itemize}[nosep]
    %   \item Created a {\bf C++/Arduino}/PlatformIO system for real-time music data analysis and visualizations
    %   \item Designed a {\bf Python} music visualization tool for prototyping analysis \& visualization algorithms
    % \end{itemize}

    % \href{https://www.roblox.com/games/272941/Robloxaville}{\bf Robloxaville}
    % \vspace{-6pt}
    % \begin{itemize}[nosep]
    %   \item Remastered a popular {\bf Lua} game on the ROBLOX game platform supporting both PC \& mobile gameplay
    %   % \item Engineered project to patch security flaws and emphasize project maintainability and scalability 
    % \end{itemize}
    
%    {\bf Google Assistant Transit Tracker}
%    \hfill {\em HackTheU 2018}
%    \vspace{-6pt}
%    \begin{itemize}[nosep]
%       % \item Created a Google DialogFlow application to retrieve local bus schedules via Google Assistant
%      \item Parsed and converted Google DialogFlow voice commands to {\bf SQL} queries of UTA schedule databases
%      % \item Deployed application back-end to {\bf Firebase} in order to run all back-end code on the cloud
%    \end{itemize}

\end{rSection}

% \begin{rSection}{Activities}

    % {\bf Vice President}, {\em Triangle Engineering Fraternity \hfill April 2020 - Current}
    % \vspace{-6pt}
    % \begin{itemize}[nosep]
    %     \item Set up and administrate a growing Discord server of $>$500 students and faculty at the University of Utah which provides academic, professional, and social resources for members of the fraternity and community.
    %     \item Work in a team of three to develop features for a Discord bot written in TypeScript for the server, including automated class enrollment and verification of student/faculty status.
    %     % \item Teach professional/personal development curriculum for an organization of roughly 40 members
    %     % \item Designed and presented
    %     %     \href{https://docs.google.com/presentation/d/1fGTdp4_FkZzHyAAKyMnpKdpii130kyLaFquVrfw7vdY/edit?usp=sharing}
    %     %          {improvements on management style}
    %     %     to cut logistical meetings by 75\%
    % \end{itemize}
    
    % {\bf Director of Marketing}, {\em HackTheU \hfill January 2020 - Current}
    % \vspace{-6pt}
    % \begin{itemize}[nosep]
    %     \item Direct the marketing division of the committee that runs Utah's largest hackathon in preparation for a small virtual hackathon in March 2020 and a large virtual hackathon in October 2020.
    %     \item Create a strong digital marketing platform with connections to media outlets throughout the Salt Lake Metropolitan area and companies such as Galileo, esri, L3Harris, and MLH.
    % \end{itemize}
% \end{rSection}

%-----------------------------------
%
%   REFERENCES
%
%-----------------------------------
\vspace{8pt}
\begin{center}
    {\em References upon request}
\end{center}

\end{document}

%-----------------------------------
%
%   METADATA
%  
%-----------------------------------
\hypersetup{
    pdftitle={Jared Amen Resume},
    pdfsubject={Software Engineering, Natural Language Processing, Social Computing},
    pdfauthor={Jared Amen},
    pdfkeywords={

    
                }
}